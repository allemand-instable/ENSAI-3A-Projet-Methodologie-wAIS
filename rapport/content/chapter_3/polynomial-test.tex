
\subsubsection{Monte Carlo Method for a known Polynomial}


We consider a function which has an easy anti-derivative to compute such as :


$$P(x) = ax^3 + b x^2 + c x$$

we decompose the function as $f = h \times q $ to compute it using Mote Carlo methods :

$$\begin{array}{rcl}
\displaystyle\int\limits_A^B P(x)dx 
&=& 
\displaystyle\int\limits_A^B \underbrace{\sqrt{2\pi} e^{\frac{x^2} 2}P(x)}_{h} \underbrace{\frac{e^{\frac{-x^2} 2}}{\sqrt{2\pi}}}_{q = \mathcal N(0,1)} \ dx
\\
&=& \esperanceloi{\mathcal N(0,1)}{h(X)} 
\\
&=& \sqrt{2\pi} \esperanceloi{\mathcal N(0,1)} {e^{\frac{X^2} 2}P(X)}
\end{array}$$

However we know the integral precisely :

$$
\begin{array}{rcl}
\displaystyle\int\limits_A^B P(x)dx 
&=& 
\left[ \frac a 4  x^4 + \frac b 3  x^3 + \frac c 2  x^2 \right]^B_A \\
&=& \frac a 4 \left[ B^4 - A^4 \right] + \frac b 3 \left[ B^3 - A^3 \right] 
\\ & &
\\ & &+ \frac c 2 \left[ B^2 - A^2 \right]
\end{array}
$$

We will first start using a sampling policy $q_0 = \mathcal{N}(\mu_0, \sigma_0)$ and compare the results from the weighted Normalized AIS to the precise evaluation of the integral in order to make sure the algorithm works properly before doing any benchmark.
