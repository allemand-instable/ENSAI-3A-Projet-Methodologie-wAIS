\subsection{Other divergence measures}

\subsubsection{Rényi's Alpha Divergence}

$$
\begin{array}{rcl}
	R_\alpha(p \Vert q) &=& \frac 1 {\alpha - 1} \log \esperanceloi q { \left( \frac {p(X)} {q(X)} \right)^\alpha }
\\
R_\alpha(q_\theta \Vert f) &=& 
\frac 1 {\alpha - 1} \log \esperanceloi f { \left( \frac {q_\theta(X)} {f(X)} \right)^\alpha }
\\
&=& 
\frac {1}{\alpha - 1} \log \esperanceloi{ q_0 }{ \left( \frac{f(X)}{q_0(X)} \right) \left( \frac {q_\theta(X)} {f(X)} \right)^\alpha }
\end{array}
$$

therefore we can derive the following expression of the gradient of the Rényi's alpha divergence, using the differentiation under the integral theorem :


$$\nabla_\theta R_\alpha(q_\theta \Vert f) = \frac {1}{\alpha - 1} \log \esperanceloi{ q_0 }{ \left( \frac{f(X)}{q_0(X)} \right) \nabla_\theta \left( \frac {q_\theta(X)} {f(X)} \right)^\alpha }$$

we can deduce from there, using the law of large numbers an estimator of $\nabla_\theta R_\alpha(q_\theta \Vert f)$, which will be noted as $\widehat{\nabla R_\alpha}$:

\[
	\begin{array}{rcl}
		
	\widehat{\nabla R_\alpha} &=& \displaystyle{\frac 1 N \sum\limits_{i=1}^N  \left( \frac{f(x_i)^{1-\alpha}}{q_0(x_i)} \right) \widehat{\nabla_\theta }\left[\left( {q_\theta(x_i)} \right)^\alpha\right]}
		\\
	&=& \displaystyle{\frac 1 N \sum\limits_{i=1}^N \omega_i \cdot h_i(\theta)}
\end{array}
\]

	with :
	\begin{itemize}
	
		\item $\displaystyle{\omega_i \isdef \frac{f(x_i)^{1-\alpha}}{q_0(x_i)}}$
		\item $\displaystyle{h_i(\theta) \isdef \widehat{\nabla_\theta }\left[\left( {q_\theta(x_i)} \right)^\alpha\right]}$
	\end{itemize}


\subsubsection{Amari's Alpha Divergence}

$$
	\begin{array}{lcl}
		A_\alpha( p \Vert q ) & = & \frac{1}{\alpha( \alpha - 1 )} \left[ \int p^\alpha q^{1 - \alpha} d\lambda \ -1 \right]
		\\
		                      & = & \frac{1}{\alpha( \alpha - 1 )} \left( \esperanceloi q { \left( \frac {p(X)} {q(X)} \right)^\alpha } - 1\right)
	\end{array}
$$

we can use ideas from importance smapling to derive the following expression :


\[
	\begin{array}{lcl}

		A_\alpha( p \Vert q ) & = & \frac{1}{\alpha( \alpha - 1 )} \left( \esperanceloi q { \left( \frac {p(X)} {q(X)} \right)^\alpha } - 1\right)

		\\
		                      & = & \frac{1}{\alpha( \alpha - 1 )} \left( \esperanceloi {q_0} { \left( \frac {p(X)} {q(X)} \right)^\alpha  \left( \frac{q(X)}{q_0(X)} \right) } - 1\right)
		\\
	\end{array}
\]


And therefore, we have :

$$A_\alpha( p \Vert q_\theta ) = \frac {\esperanceloi {q_0} { \omega(X \vert \theta) \cdot r_\alpha(p \Vert q_\theta )(X) } - 1 }{\alpha ( \alpha - 1 )}$$


with :
\begin{itemize}
    \item $r_\alpha( p \Vert q_\theta ) = \left( \frac {p(X)} {q(X)} \right)^\alpha$
    \item $\omega( X \vert \theta ) = \frac{q_\theta(X)}{q_0(X)}$
\end{itemize}

The gradient of the divergence can therefore be expressed as (assuming the conditions to invert derivative and integral are met):

$$
\nabla_\theta A_\alpha( p \Vert q_\theta ) = 
\frac {
\esperanceloi {q_0} { \nabla_\theta 
\left[\omega(X \vert \theta) \cdot d_\alpha(p \Vert q_\theta )(X)\right] } 
}
{\alpha ( \alpha - 1 )}
$$
